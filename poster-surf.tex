\documentclass[final,hyperref={pdfpagelabels=false}]{beamer}
\usepackage{grffile}
\mode<presentation>{\usetheme{I6pd2}}
\usepackage[english]{babel}
\usepackage[latin1]{inputenc}
\usepackage{amsmath,amsthm, amssymb, latexsym}
%\usepackage{times}\usefonttheme{professionalfonts}  % obsolete
%\usefonttheme[onlymath]{serif}
\boldmath
\usepackage[orientation=portrait,size=a0,scale=1.4,debug]{beamerposter}
% change list indention level
% \setdefaultleftmargin{3em}{}{}{}{}{}


%\usepackage{snapshot} % will write a .dep file with all dependencies, allows for easy bundling

\usepackage{array,booktabs,tabularx}
\newcolumntype{Z}{>{\centering\arraybackslash}X} % centered tabularx columns
\newcommand{\pphantom}{\textcolor{ta3aluminium}} % phantom introduces a vertical space in p formatted table columns??!!

\listfiles

%%%%%%%%%%%%%%%%%%%%%%%%%%%%%%%%%%%%%%%%%%%%%%%%%%%%%%%%%%%%%%%%%%%%%%%%%%%%%%%%%%%%%%
\graphicspath{{figures/}}

\setlogo{figures/deselaers/logos/https://github.com/AnqiAngelineWang/SOEN6011-yelloworange}
\setauthorurl{}
\setauthoremail{}

\title{\huge SOEN 6011 Project Function 8 Beta(x, y)}
\author{Anqi Wang 40057695 Github Repo: https://github.com/AnqiAngelineWang/SOEN6011}
\institute[Concordia University]{Software Engineering, Concordia University, Montreal, Canada}
\date[Aug. 12th, 2019]{Aug. 12th, 2019}

%%%%%%%%%%%%%%%%%%%%%%%%%%%%%%%%%%%%%%%%%%%%%%%%%%%%%%%%%%%%%%%%%%%%%%%%%%%%%%%%%%%%%%
\newlength{\columnheight}
\setlength{\columnheight}{105cm}


%%%%%%%%%%%%%%%%%%%%%%%%%%%%%%%%%%%%%%%%%%%%%%%%%%%%%%%%%%%%%%%%%%%%%%%%%%%%%%%%%%%%%%
\begin{document}
\begin{frame}
  \begin{columns}
    % ---------------------------------------------------------%
    % Set up a column 
    \begin{column}{.49\textwidth}
      \begin{beamercolorbox}[center,wd=\textwidth]{postercolumn}
        \begin{minipage}[T]{.95\textwidth}  % tweaks the width, makes a new \textwidth
          \parbox[t][\columnheight]{\textwidth}{ % must be some better way to set the the height, width and textwidth simultaneously
            % Since all columns are the same length, it is all nice and tidy.  You have to get the height empirically
            % ---------------------------------------------------------%
            % fill each column with content            
            
            
         %%%%%%%%%%%%%%%%%%%%%%%%%%%%%%%%%%%%%   
            \begin{block}{Introduction and Description}
              \begin{itemize}
              \item Beta function B(x,y)is the incomplete gamma functions $ \gamma (a,z) $ and $\Gamma(a,z)$. It is one of the important meaningful mathematic functions. Generally speaking, Beta function is related with Euler Integral and is the first kind. It can be considered as the incomplete beta functions. Beta function has the general form: \\

 \State $ B (x,y) =  \Gamma(x) \Gamma(y)  / \Gamma( x+y) $ \\
 \State $ B (x,y) =  $$\int_{0}^{\infty} t^{x-1} \times (1-t)^{y-1} dt$$  $ \\ \\

                \begin{itemize}
                \item Domain: x,y $ \subset (0, + \infty)$  for all real value, greater than 0. \\
Co-domain: the solution generated by B(x,y) , satisfy with  x,y $ \subset (0, + \infty)$ 

                \end{itemize}
              \item It has the shape:
                \begin{itemize}
                
\begin{figure}[h]
\begin{center}
\includegraphics[width=120mm,scale=1]{1} 
\end{center}
\end{figure}

            Figure 1. Graph of the Beta Function   
                
                \end{itemize}
              \end{itemize}              
            \end{block}
            \vfill
       %%%%%%%%%%%%%%%%%%%%%%%%%%%%%%%%%%%%%        
           
             
            \begin{block}{Functional and Non-functional Requirement}
              \begin{itemize}      
  \item   These functional and non-functional requirements followed ISO/IEC/IEEE 29148 Standard. Their individual rationale with detailed explanation are in the report. 
            
             
              \item Functional Requirement
                            
\begin{figure}[h]
\begin{center}
\includegraphics[width=120mm,scale=1]{8-8} 
\end{center}
\end{figure}
    
     \item Non-functional Requirement
                
              \begin{figure}[h]
\begin{center}
\includegraphics[width=120mm,scale=1]{3} 
\end{center}
\end{figure} 
                \begin{itemize}
    
                \end{itemize}
               \end{itemize}
            \end{block}
            \vfill
            
           
           
            %%%%%%%%%%%%%%%%%%%%%%%%%%%%%%%%%%%%%        
           
           
           
           
         
         
            \begin{block}{Algorithm Selection}
            \begin{itemize}
           
              
             
              \item Algorithm Optional 1
                \begin{itemize}
                \item Explanation: This algorithm works based on approximation values in the array to estimate beta results.
                \item Advantages: Easy to understand the logistic behind. \\
Disadvantages: This algorithm can only obtain accurate results for integer inputs, not for decimal inputs. The result has large uncertainties.


                \end{itemize}
               %%%%%%%%%%%5 
                 \item Algorithm Optional 2
                \begin{itemize}
                \item Explanation:This algorithm generates beta results based on mathematic models.
                \item Advantages: The code structure is clear and easy to implement. \\
Disadvantages: This algorithm needs to use built-in functions for mathematical processing, like Math.floor, or Math.exp.




                \end{itemize}
                
                
                
                
                
                %%%%%%%%%%%%
                   \item Algorithm Optional 3
                \begin{itemize}
                \item Explanation:This algorithm calculate Beta solutions based on input value ranges. It has accuracy for almost 15 digits after decimal point. 
                \item Advantages:  It has high accuracy and low errors in results. It can handle both integers and decimals inputs. \\
Disadvantage: It needs lightly longer processing time.




                \end{itemize}
                
                  %%%%%%%%%%%%
                   \item Algorithm Optional 3 has been selected for the implementation.


                
                
                %%%%%%%%%%%%
        
                
                
      
              \end{itemize}
            \end{block}
            \vfill
  
  
  
  %%%%%%%%%%%%%%%%%%%%%%%555
         
            \begin{block}{Reference}
            \begin{itemize}
           
              
             
              \item      Stewart, J. (2008). Transcendental Functions [Abstract]. Calculus, 6, 71-73. Retrieved August 12, 2019.
              
                     \item ISO/IEC/IEEE 29148:2018. Systems and Software Engineering -- Life Cycle Processes -- Requirements Engineering. 2019. ISO, IEC, & IEEE. (P.9-P.16)
\\

\item Olver, F. W., Lozier, D. W., Boisvert, R. F., & Clark, C. W. (Eds.). (2010). NIST handbook of mathematical functions hardback and CD-ROM. Cambridge university press.

      
              \end{itemize}
            \end{block}
            \vfill
  
            
            %%%%%%%%%%%%%%%%%%%%%%%%%%
           
            %%%not use %%%%
            
          }
        \end{minipage}
      \end{beamercolorbox}
    \end{column}
    % ---------------------------------------------------------%
    % end the column

    % ---------------------------------------------------------%
    % Set up a column 
    \begin{column}{.49\textwidth}
      \begin{beamercolorbox}[center,wd=\textwidth]{postercolumn}
        \begin{minipage}[T]{.95\textwidth} % tweaks the width, makes a new \textwidth
          \parbox[t][\columnheight]{\textwidth}{ % must be some better way to set the the height, width and textwidth simultaneously
            % Since all columns are the same length, it is all nice and tidy.  You have to get the height empirically
            % ---------------------------------------------------------%
            % fill each column with content
            
            \begin{block}{Implementation}
            
                    \begin{itemize}
          {\small            \item  This source code has followed the standard Google Java Style Guide, which
is corresponding to the whole team's program style. The team has confirmed
that source code file structure is correspondence. }
               {\small       \item Correctness and Efficiency: For this program, the maximum length
allowance number is followed double data type 1.79E308. Double datatype is a primitive datatype and it
does not require much system memories. }
                 {\small     \item Maintainability and Program Style: For this program, coding convention has been regulated. corresponding team's program style has kept with
the same format, block indentation and statement has been regulated. The
code follows line wrapping and line break strictly. There is no extra whites-
pace. Exactly each section has one blank line to separate. }
 {\small 
\item Debugger: Jetbrains InteliJ Idea IDE build-in debugger is the major tool.
\item Checkstyle: This program uses Checkstyle devel-
opment tool during software implementation.
}
 {\small 
\item Error handling and User Interface: This program has a clear user interface. It is straight forward for user to un-
derstand the logic and instructions. This program also kindly reminds user if
he has input a valid value. If error occurs, error handling exceptions shall be
invoked. Try and catch blocks keep the program functioning, and exceptions
have thrown error messages to remind user. The figure below shows the result.
}
              
\begin{figure}[h]
\begin{center}
\includegraphics[width=120mm,scale=1]{4} 
\end{center}
\end{figure}

            Figure 2. User Interface with Exception Handling, Error Messages   
    {\small              
\item The above run cases shows that functional requirements has been processed, especially for boundary check.
}



                    \end{itemize}
              
            \end{block}
            \vfill
            
            
            
            %%%%%%%%%%%%%%%%%%%%%%%
            \begin{block}{Unit Testing}
              
              
              \begin{itemize}
              
        {      \item  This program introduces Junit assert .java file (AssertTests.java). The test cases have satisfied client's requirement, and matches with user assumptions. The program has passed all test cases. 
Test cases ID and user (assumptions) requirements ID have matched, and explain below:
}
            
\begin{figure}[h]
\begin{center}
\includegraphics[width=120mm,scale=1]{5} 
\end{center}
\end{figure}

            Figure 3. Each test case contains requirement 
            
                       
\begin{figure}[h]
\begin{center}
\includegraphics[width=120mm,scale=1]{6} 
\end{center}
\end{figure}
              
            Figure 4. All test cases have passed
              
              
         {\small       
         \item     Software testing is believed to be useful to check program accuracy, satisfy requirements, and increase software robustness.
}
              \end{itemize}
                    
            \end{block}
            \vfill
            
            %%%%%%%%%%%%%%5
            
 \begin{footnotesize}           
            \begin{block}{Conclusion and Reflects}
           
            
                  \begin{itemize}
            
            \item Test cases almost covered majority of source code. 
                                   
\begin{figure}[h]
\begin{center}
\includegraphics[width=120mm,scale=1]{7} 
\end{center}
\end{figure}
              
            Figure 5. Team member code review result
           {\small   \item  Critical decisions: When I decide which algorithm to choose, it is the critical decision. There are algorithms are accurate, but not satisfying the project requirement, like it needs to use math packages to simulate integral calculation. Or, the algorithm is not precise for this project. }
              
             {\small   \item Lesson learnt myself:  I need to make clear for functional and non-functional requirement. And test cases need to match with requirement. Also, test cases had better cover most of the codes, which will be more persuaded to demonstrate the program's testability. }
              
           {\small     \item Lesson learnt from team member: I need to pay attention in coding format. The format I followed must be correct and accurate, especially, no extra spaces in the source code. Also, I need to pay attention to naming convention. This is important for readability. While coding, line length had better not exceed 100 characters. Last but not least, writing exception handling error message should be enhanced. This will provide user easier time to understand the problem happened in the code. }
   
                  \end{itemize}
        
            \end{block}
            \end{footnotesize}
            \vfill
            
            %%%%%%%%%%%%%%%%%%
         
            
            
          }
          % ---------------------------------------------------------%
          % end the column
        \end{minipage}
      \end{beamercolorbox}
    \end{column}
    % ---------------------------------------------------------%
    % end the column
  \end{columns}
  \vskip1ex
  %\tiny\hfill\textcolor{ta2gray}{Created with \LaTeX \texttt{beamerposter}  \url{http://}}
  \tiny\hfill{Created with \LaTeX \texttt{beamerposter}  \url{} \hskip1em}
\end{frame}
\end{document}


%%%%%%%%%%%%%%%%%%%%%%%%%%%%%%%%%%%%%%%%%%%%%%%%%%%%%%%%%%%%%%%%%%%%%%%%%%%%%%%%%%%%%%%%%%%%%%%%%%%%
%%% Local Variables: 
%%% mode: latex
%%% TeX-PDF-mode: t
%%% End:

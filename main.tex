\documentclass{article}
\usepackage[utf8]{inputenc}

\title{SOEN 6011 Problem 2 Anqi Wang Team K }



\begin{document}
\author{F8 B(x, y)}
\maketitle
{GitHub AnqiWang's individual repo address: https://github.com/AnqiAngelineWang/
SOEN6011.git
}



\section{Description}
Beta function B(x,y)is the incomplete gamma functions $ \gamma (a,z) $ and $\Gamma(a,z)$. It is one of the important meaningful mathematic functions. Generally speaking, Beta function is related with Euler Integral and is the first kind. It can be considered as the incomplete beta functions $B (a,b) =  \Gamma(x) \Gamma(y)  / \Gamma( x+y) $. 
The general requirement to use Beta function B(x, y) needs real values for x and y.  \\


Domain: x,y $ \subset (0, + \infty)$  for all real value, greater than 0. \\
 \ \ \ \ \ \ \ \ Co-domain: the solution generated by B(x,y) , satisfy with  x,y $ \subset (0, + \infty)$ 
 
\section{Characteristic}
Beta function is good for generating cumulative results. The result is based on a random variable in the binomial distribution which can be obtained from beta distribution.  As for the Beta function putting into discussion, the regularized incomplete beta function is aim to establish the "probability of success", based on the input variable size. 

\section{Assumption}
1.	User has input a character \\
2.	The input number is a negative infinite decimal number \\
3.	The input number is a complex number, also a real number \\
4.	User has input a very large number that has out of the machine memory \\
5.	User has input a big number that the system couldn’t handle \\
6.  User has modified the input again after his first input\\ 

\section{Requirement}
 
1.	The system shall respond that the input type is invalid \\
2.	The system shall respond that the input number is not in the Beta Function’s acceptable range \\
3.	The system shall respond that the system couldn’t calculate complex number, only real number \\
4.	The system shall respond that this system can not support for this calculation  \\
5.	The System shall overflow \\
6.  The System shall calculate based on the user's first input \\

\section{Explanation}
Putting Beta function into software engineering as artifacts, it should be uniquely identified. Requirement engineers will discover the stakeholder requirement or system requirement in the aspects that users or acquirers are satisfied with. If the assumption to put into Beta Function is not user’s requirement, then the result is not useful. During the design, it needs to make sure that the work has its version number corresponding to its owner, and has not been reused. Requirement engineering will be introduced to verify the system’s reliability, feasibility, validity and if it is able to be implemented. They also need to distinguish the assumptions and constraints. For example, the output result of Beta Function should be a real number and needs to pass test cases. This is the risk that requirement engineering needs to handle before put into practice. Also, the schedule of designed implementing time, cost and applied technology is out of control as an example, it may lead to failure. Thus, it should have a valid tracking documentation, like recording steps to save procedures, and providing traceable evidence. \\
Besides, stakeholders can use Beta function to estimate the probability of the product’s properties. This primary result may lead users to an efficient way rather than fumbling future. It gives logistic and reasonable suggestions for stakeholders to make decisions on their choices. For example, users will apply the beta function on their products. The results may influence their next investment aspects. It is also easier for them to forecast their future developments. \\


 \section{Reference}
 
\ \ \ \ \ \ ISO/IEC/IEEE 29148:2018. Systems and Software Engineering -- Life Cycle Processes -- Requirements Engineering. 2019. ISO, IEC, & IEEE. (P.9-P.16)
\\

Olver, F. W., Lozier, D. W., Boisvert, R. F., & Clark, C. W. (Eds.). (2010). NIST handbook of mathematical functions hardback and CD-ROM. Cambridge university press.




\end{document}

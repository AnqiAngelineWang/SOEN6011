
\documentclass{article}
\usepackage{graphicx} % Required for the inclusion of images
\usepackage{natbib} % Required to change bibliography style to APA
\usepackage{amsmath} % Required for some math elements 

\setlength\parindent{0pt} % Removes all indentation from paragraphs

\renewcommand{\labelenumi}{\alph{enumi}.} % Make numbering in the enumerate environment by letter rather than number (e.g. section 6)

% environment
\begin{small}

{SOEN 6011 \ Problem 1 \ Anqi Wang 40057695 \  F8 B(x, y) \\ GitHub Anqi Wang's individual repo address: https://github.com/AnqiAngelineWang/ \\ SOEN6011.git \\ \ GitHub group repo address: https://github.com/SinglaAnkur/SOEN6011.git  } \\
\end{small}



\begin{document}

\begin{small}
{Explain the function B(x,y)}
\end{small}


\begin{small}
B(x,y) is a function contains two real variables x and y. It could not be shown in a finite sequence among algebraic operations. Thus, it will not provide a specific answer, like it will never receive a fix answer 0. The formation of B(x,y) are varies based on x and y variables. It may or may not be a polynomial equation. \\
\end{small}

\begin{small}
{Domain: Any set of rational number } 
\end{small}

\begin{small}
{Co-domain: }Any real number $(- \infty, + \infty)$ \

\end{small}

\begin{small}
{Characteristics}
\end{small}


This function needs two variables x and y to come up with a solution. The possibilities of the function results are various. It may not be symmetry in a specific range, but it could also possibly show the monotone increasing trend in a long period. What’s more, it can be considered as a set (x, y), or a coordinate point (x, y) on a 2D graph. For example, there is a unit circle in the xy-plane, the graph can be like:  

\begin{figure}[h]
\begin{center}
\includegraphics[width=0.65\textwidth]{Problem1_figure} 
\end{center}
\end{figure}



The x and y are angles of the circle centre. The coordinate $ A( \cos x , \sin x) $ are functions $  \cos x $ and $  \sin x $. This reveals that the function may not need to follow monotone trend. Besides, the answer can also be obtained through mathematics triangle definitions: $ (\sin x )^{2} + ( \cos x )^{2} = 1 $, such that the hypotenuse of the unit triangle is 1.
\\


\begin{small}
{Reference}
\end{small}

\begin{small}
Stewart, J. (2008). Transcendental Functions [Abstract]. Calculus, 6, 71-73. Retrieved July 7, 2019.
\end{small}



\end{document}